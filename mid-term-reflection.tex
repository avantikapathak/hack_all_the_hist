\documentclass[12pt]{article}
\usepackage{graphicx}

\begin{document}

\title{Hackerspaces on the West Coast and in India \\ Mid-Term Reflection}
\author{Kevin Vilbig}
\maketitle

I've been in school for a long time. I've spent time talking with professors and have taken their advice to heart. The most important advice that I've gotten over the years hasn't necessarily been in the classroom per se.  That said, the most important thing as a researcher is to realize that the project that you plan might not be the project that you submit. I had a history professor at University of Texas at Dallas whose personal research at that time was the result of a completely unexpected turn. She went to the archives in Florence for one purpose that I cannot remember, but ended up becoming fascinated with some letters she found that were correspondence between Michalangelo and a woman whose name escapes me currently.  I also have forgotten the name of that professor.  I am hoping to figure it out (via an e-mail with their department) because she was one of my great guides through my academic career. I've also run into notes in the prefaces of a number of texts that I've read that suggest that almost always, honest academic projects take unexpected turns. Our preconcieved notions of what we're going to find almost always smack head-first into the wall of reality, and our job is, ultimately, to create some kind of real knowledge from the experiences that we have, even if, especially if, we are led down some strange paths on the way to that goal.

Regarding being on track? Honestly, I don't even know if I'm on track or not. My goals are to craft an account of the experiences that I have in looking for the Indian hacker community. I thought that was going to be centered around these hackerspaces, or at least the genesis of the exploration would begin there, but instead my first experience was interacting with some Indian con-men, who were grifting me for money and info for a couple of days. In the popular sense of the word hacker, they showed me some Indian video-streaming websites that had what was probably a pirated version of a brand new Indian movie. They also had a very convincing office and were doing a good job of running a long-con on me. I'm starting to realize, that as a participant observer/ethnographer, sometimes just finding the culture that you are trying to study can be a learning experience in-itself. This difficulty is compounded because of traveling in a foreign culture, looking for a sub-culture within the culture. It is difficult to get around the city on my own volition. I've also figured out exactly where this house I am looking for is located, finally. It's a little more like a commune/shared living space that caters to techies than anything else. I'm beginning think they are a group of well meaning Indian young people who want to focus on social-justice, and think that technology is a way to do that. Which is exactly what I am seeking here. I'll go there tomorrow.

I'm also starting to see patterns in this microcosm of Indian society where I have been living. I'm staying in the Main Bazar in New Delhi. Sikhs and Hindus and Bahá'í and Muslims and Catholics and Jainists and... all live together in extremely close proximity, all with different values and motivations. It's the same kind of pattern with different parts of the city. Residential complexes are very large (on the order of square kilometers) and very separate from the bazars where there are shops and hotels and the poor people. There are walls and fences everywhere and in-between, the mass of people.  Most have cellular phones, which seem to be the main mode of telecommunications here in India. Voice service is very inexpensive here compared to the US, but the Indian Government requires every line to be attached to someone via a photo ID. So, it is illegal to have anonymous communications.  There are a dizzying array of cyber cafe's and mobile retailers. It might be a good idea to check them out and talk to some people there to see how normal people use the internet. The young desk clerk at my hotel was using Facebook, so it might be worth it to talk to him as well.

Mostly, I realized exactly how much of an endeavor this really is.

\end{document}